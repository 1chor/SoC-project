The presented implementation is far from perfect.
Issues faced during development that could not be fixed during the course of the
project are presented in this section.
\subsection{Android Version}\label{ssec:issueandroidversion}
Android Gingerbread (2.3.7) was released on December 6\textsuperscript{th} 2010,
more than 8 years ago at the time of writing this document.
9 major versions were released since then.
Unfortunately, the ZedBoard does not have the hardware requirements for more
recent version (little RAM and no dedicated \gls{gpu}).
Although, the higher-range Zynq chips do have the requirements to run the most
up to date versions and Mentor Embedded even provides support for Android 6
(Marshmallow), 7 (Nougat) and 8 (Oreo).
The same methodology as we presented could be employed on those higher range
devices.
The Mentor Embedded distribution was tested on the Zynq zcu102 developement
board and was proven to be working.
That distribution does not include a hardware design (the bitstream is
distributed in binary format) and therefore not allow to alter the logic fabric
on the \gls{fpga}.
One could create a custom hardware design, but that task was deemed to be too
time consuming considering the deadline for this project.

Noritsuna Imamura managed to get Android 5.0 running on the
ZedBoard~\cite{noritsuna}.
A linux swap partition was used to extend the RAM.
Noritsuna states that his system is very slow.
This is most probably due to the fact that it still uses software rendering and
that the swap partition lies on the SD card and is therefore very slow.
Nevertheless we tried to replicate his project but did not manage to do so.
All the parts build without errors, but the system does not boot.
Noritsuna has not configured the kernel to have a terminal console over the
\gls{uart} interface, so we had no lead on where to start debugging.

\subsection{Touchscreen Input}\label{ssec:issuetouchscreen}
The touchscreen is not working in the current design.
After configuring the kernel to support single- and multitouch devices and
creating a \gls{idc}, the device is detected in Linux and input-events are sent
to android.
Due to unknown reasons, the events are never processed by the Android \gls{gui}.

We observed the following:
\begin{itemize}
	\item The device is recognized with all it's features by Android's built-in
		\emph{getevent} utility
	\item Events are registered by Android's built-in \emph{getevent} utility
	\item The translation from touch- to screen-coordinates works as designed
	\item Android's \emph{InputReader.cpp} detects the device, creates an
		appropriate event handler and reacts to events
	\item When configured as multi-touch device, the events are not dispatched
		to the \gls{gui}, because of the touchscreen never sending
		\emph{SYN\_MT\_REPORT} messages.
		This could be amended by patching the \emph{InputReader.cpp} to react to
		\emph{KEY\_MT\_SLOT} messages instead.
		This was not tried because it seems like even if the events are
		dispatched, they are not picked up by the \gls{gui} (see next point).
	\item When configured as single-touch device, the events are dispatched to
		the \gls{gui}, but for unknown reasons they are never picked up.
		They are handed over from native code to Android's Dalvik Java \gls{vm}
		but never make it to the Java Input Queue in\\
		(\emph{<repo>/android/frameworks/base/core/java/android/view})
	\item We built \emph{tslib}, a cross-platform library that provides access
		to touchscreens for Android.
		This library contains a calibration utility.
		That utility kills the Android \gls{gui} and creates a minimal \gls{gui}
		by itself.
		Even when started from Android, the utility recognizes all touches
		correctly.
		This suggests the issue lies within how the Android \gls{gui} reacts to
		events and that the touchscreen driver is working fine
	\item Android Gingerbread has support for mice connected via \gls{otg}.
		We observed that this is not working either.
		The behavior is much the same as with the touch-screen in single-touch
		configuration.
		Events are detected but not handled by the \gls{gui}.
\end{itemize}
